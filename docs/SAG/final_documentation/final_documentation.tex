\documentclass[a4paper,11pt]{article}
\usepackage[utf8]{inputenc}
\usepackage[MeX]{polski}
\usepackage{hyperref}
\usepackage{graphicx}
\usepackage{pdfpages}
\usepackage{tikz}
\usepackage{fancyvrb}
\usepackage{rotating}
\usepackage{multirow}
\usepackage{amsmath}
\usetikzlibrary{positioning,shapes,shadows,arrows}

\author{Michał Aniserowicz \href{mailto:michalaniserowicz@gmail.com}{{\small \nolinkurl{<michalaniserowicz@gmail.com>}}} \\ Jakub Turek \href{mailto:jkbturek@gmail.com}{{\small \nolinkurl{<jkbturek@gmail.com>}}}}
\title{{\Large [SAG.A] Dokumentacja końcowa projektu} \\ Protokół MAS z~wykorzystaniem ontologii}
\date{2 czerwca 2013r.}

\begin{document}

\maketitle

\section{Temat projektu}

Tematem projektu jest implementacja protokołu wykorzystującego ontologię w~systemie wieloagentowym. W ramach projektu opracowana została symulacja ruchu drogowego w~kwadratowej sieci ulic przypominającej plan Manhattanu. Symulacja składa się z~następujących elementów:

\begin{description}
    \item[miasto] określa ilość przecznic (wielkość przestrzeni),
    \item[kodeks ruchu drogowego] definiuje zasady poruszania się na drodze,
    \item[sygnalizacja świetlna] określa pierwszeństwo na części skrzyżowań,
    \item[samochód] porusza się po mieście zgodnie z~zasadami ruchu drogowego znajdującymi się w~kodeksie.
\end{description}

Projekt obejmuje również przygotowanie graficznego interfejsu użytkownika, który pełni dwojaką funkcję:

\begin{itemize}
    \item ukazuje aktualny stan miasta w~rzucie z~góry,
    \item pozwala na dynamiczną zmianę kodeksu ruchu drogowego.
\end{itemize}

\section{Technologia}

Projekt został zaimplementowany na platformie JADE\footnote{Java Agent DEvelopment Framework - \href{http://jade.tilab.org}{jade.tilab.org}.}. Interfejs graficzny został stworzony z~użyciem natywnych bibliotek języka Java (AWT oraz Swing). Implementacja tworzona była w~środowisku Eclipse i~testowana w~systemie operacyjnym Windows 7 64-bit.

\end{document}