\documentclass[a4paper,10pt]{article}
\usepackage[utf8]{inputenc}
\usepackage[MeX]{polski}

\title{[MBI.A] Asembler DNA, sparowane końce - dokumentacja wstępna}
\author{Michał Aniserowicz, Jakub Turek}
\date{}

\begin{document}

\maketitle

\section*{Opis problemu}

Zadanie polega na implementacji aplikacji, która umożliwia tworzenie \emph{scaffoldów} na podstawie dostarczonych zbiorów \emph{contigów} oraz sekwencji PET. 

\begin{verbatim}
  <insert drawing in here>
\end{verbatim}

\section*{Założenia}

W~ogólnym przypadku rekonstrukcja sekwencji \emph{contigów} nie jest możliwa. Z~tego względu, na potrzeby projektu przyjęto następujące założenia:

\begin{itemize}
  \item Początki i~końce łańcuchów PET to sekwencje unikalne. Wystąpienie takiej sekwencji w~jednym z~\emph{contigów} oznacza, że jest to odpowiednio początek lub koniec sekwencji PET.
  \item Badane są wyłącznie takie permutacje \emph{contigów}, dla których wystąpienie początku sekwencji PET implikuje przynajmniej częściowe wystąpienie jej końca w~dalszej części łańcucha. Innymi słowy początek lub koniec sekwencji PET nie może w~całości wystąpić w~przerwie (\emph{gap}) \emph{scaffoldu}.
  
    \begin{itemize}
	  \item Wyjątkiem od tej reguły jest początek i~koniec sekwencji, gdzie mogą występować, odpowiednio, niesparowane końce lub początki sekwencji PET.
    \end{itemize}
  
  \item Sekwencje należące do różnych par sparowanych końców mogą częściowo zachodzić na siebie.
\end{itemize}

\section*{Algorytm}

Do rozwiązania zadania użyty zostanie algorytm typu brute-force działający według następującego schematu:

\begin{enumerate}
 \item Wybierana jest początkowa permutacja \emph{contigów}.
 \item Dla danej permutacji obliczany jest ranking \emph{R}:
   \begin{itemize}
    \item Ranking \emph{R} określa, dla danej kombinacji \emph{contigów}, maksymalną ilość pokrywających się zasad dla zbioru dopasowań sekwencji PET do łańcucha.
   \end{itemize}
 \item Sprawdzane jest czy wartość \emph{R} jest większa niż dotychczas uzyskana maksymalna wartość rankingu. Jeżeli tak, rozwiązanie zachowywane jest jako najlepsze.
 \item Algorytm jest powtarzany dla każdej unikalnej permutacji \emph{contigów}.
\end{enumerate}

Sekwencja \emph{contigów} dobierana będzie w~sposób losowy. Jako zadanie dodatkowe może zostać przygotowana heurystyczna strategia doboru permutacji.
 
\section*{Implementacja}

Projekt zostanie zaimplementowany w~języku C\#\footnote{W~przypadku, gdy będzie to rzutowało na obniżoną ocenę (brak przenośności) projekt zostanie wykonany w~technologii Java.}. \verb+<insert technology description here>+

Aplikacja będzie posiadała interfejs okienkowy umożliwiający odczyt danych wejściowych z/zapis danych wyjściowych do pliku. Dan


\end{document}
